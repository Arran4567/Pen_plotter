\chapter{Sprint 1 - Background Research}
	\label{chap:background}
		
	Before starting the project a lot of background research is required, as there is no prior knowledge of GCODE, DXF files or safety critical systems. This chapter outlines researched topics required to gain enough understanding to begin the project and what is learnt during the research.

	\section{Safety Critical Systems}
		\label{sec:safety}

		The first section which requires research is safety critical systems. This is the main topic of the dissertation and there is no prior knowledge of the topic going into the project. This means that a lot of prior research is required in order to ensure the system follows a safety critical level design, implementation and testing.

		Safety is a not usually talked about in conventional computer science. It isn't related to reliability or security but is its own quality. Safety is the act of preventing of accidents or loss. These can come in the forms of injuries, deaths or damage to equipment. Alongside the accidents, the main concepts to consider with safety are the risks, hazards, failures, errors and faults. The risk is a combination of probability of an accident occurring and the severity of the accident should it occur. The equation $risk = p(a) * s(a)$ is usually used as a guideline for the risk of an accident, although quantifying severity can be difficult in some unforeseen cases. Hazards are the criteria that must be fulfilled for an accident to take place. When an accident occurs, the hazard is usually very minor, but minor hazards have the ability to cause accidents of major severity. A failure is when a component of the system fails randomly due to a fault. An error is when a component fails due to a predictable fault in the system. A fault is either an error or a failure.
			
			Safety is far more expensive to add to a system as an afterthought than it is to incorporate into the design and implementation phase during the entire development of the system. There is an eight-step guide to safety design and implementation:
			\begin{enumerate}
				\item Identify Hazards
				\item Determine Risks
				\item Define Safety Measures
				\item Create Safe Requirements
				\item Create Safe Designs
				\item Implement Following Safe Designs
				\item Review Previous Safety Steps
				\item Test Extensively
			\end{enumerate}
			Stages one through 3 are known as the safety analysis. These are the steps which require most forethought, as all following steps will be ensuring the chance of an accident occurring from all hazards stated in these steps are minimised. As the project is now using a simulation, instead of a robot, the hazards have already been reduced. However, all forethought will be done with an actual robot in mind, so some hazards may seem unnecessary to the final product.

			I will be addressing each of the eight stages in their respective sprint, before beginning any implementation.
			

	\section{GCODE}
		\label{sec:gcode}

		The second topic that requires research is GCODE. Prior to beginning the project, GCODE was an unfamiliar language as we had minimal knowledge of CAD. When looking through the many instructions the language consisted of, some instructions can be chosen as relevant to this project:
			\begin{itemize}
				\item G0 - Rapid Movement to specified co-ordinates. This will be useful for moving the pen when it isn't meant to be drawing. This command is used in the format: G0 X1 Y1
				\item G1 - Linear Movement to specified co-ordinates. This will be useful for moving the pen when it's meant to be drawing. This command is used in the format: G1 X1 Y1
				\item G2 - Linear movement to specified co-ordinates in a clockwise arc. The arc is centred around 2 other co-ordinates. This command is used in the format: G2 X1 Y1 I1 J1
				\item G3 - Linear movement to specified co-ordinates in a counter-clockwise arc. The arc is centred around 2 other co-ordinates. This command is used in the format: G3 X1 Y1 I1 J1
				\item M0 - Program Stop
				\item M1 - Optional Stop
				\item M2 - End of Program
				\item F - Feed Rate, or in the case of the pen plotter, Movement Speed.
			\end{itemize}
		
		There are many more GCODE instructions, however they seem unnecessary to the project at this point in time.
		
		In order to ensure the GCODE created is always created in an industry standard format we will be using a python library called pygcode.
	
	\section{DXF}
		\label{sec:dxf}

		At a basic level, DXF files are separated into features called entities. These entities define many things, such as the shape of an object, the thickness, the start and end co-ordinates, the angle at which it begins and ends, and much more.
		
		As we are only creating a 2D pen plotter, we will be minimising the number of entities we use in order to maintain simplicity and minimise errors. The entities we will use for this project will be "LINE" which is a straight line from one co-ordinate to another, "ARC" which is a curved line, "POLYLINE" which is a combination of the 2 previously stated entities and "CIRCLE" which is a 360 degree arc defined by its centre point and radius.

		In order to deal with these entities, we will be using a library called ezdxf. This will help simplify the code we produce and means that any changes in the DXF structure can be updated in the program by updating the library version.

	\section{Rasterized Images}
		\label{sec:vectorization}

		In order to covert rasterized images into GCODE, the image will first need to be converted into grayscale in order to allow for a single value to govern what is considered a desired part of the image and what is not. This will be the black value of the image, a number between 0 and 1. The user will then be given the option of drawing the contours of the image, or the entire image of black value greater than the given threshold. In order to find the contours of the image, the marching squares algorithm will be used.
		\subsection{Contours - Marching Squares}
			Marching squares is an image analysis algorithm, used to determine the contours. This is done by splitting the image into small squares and finding the black value of the corner of each square, these values are then compared to the threshold provided by the user to check if the corner is within an object or not. These squares can then be categorized into 16 unique cases, defining how the contours travel through each square. Once the case of each square is determined, a form of interpolation is required to approximate the co-ordinate at which the contour crosses the boundaries of the squares. This is usually done, and will also in the case of this project, using linear interpolation. The main issue with this algorithm is the large amount of approximation and that 2 of the cases contain ambiguity as to the orientation of the contour. This would not cause any real-world problem should the program require the objects remain close, but may cause issues when drawing the images.
		\subsection{Entire Image}
			In order to draw the entire image, an algorithm will be required to read the image row by row and convert the lines of pixels with a black value within the threshold of the user to line GCODE. This will probably be done with a self-defined algorithm.