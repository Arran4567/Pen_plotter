\section{Code Listings}
	\label{sec:typesetting_listings}
	
	Code listings should be formatted in the same style as figures and inline equations. It is important to use a monospace font so that characters line up vertically. Syntax highlighting is also extremely important for effectively displaying complicated code segments. To format inline code listings you can use the \lstinline[mathescape]|\lstinline$|$the_code$|$| command\footnote{So meta.}.
	
	% Longform Code listings should live in a code file, not embedded directly into your LaTeX code!
	\lstinputlisting[language={c}, label={lst:c_hello_world}, caption={An implementation of an important algorithm from our work.}]{./listings/hello_world.c}
	
	In LaTeX the ``Listings'' package can be used to properly format code and provide basic syntax highlighting, line numbering, and captioning of embedded code excerpts. Listing \ref{lst:listings} shows examples of how to properly format code using the listings package. 
	
	\newpage
	
	% Longform Code listings should live in a code file, not embedded directly into your LaTeX code!
	\lstinputlisting[label={lst:listings}, caption={Examples of methods for typesetting code listings within a LaTeX document.}]{./listings/listings.tex}