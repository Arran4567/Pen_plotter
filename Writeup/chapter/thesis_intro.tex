-\chapter{Introduction}
	\label{chap:intro}
	
	This thesis is intended as research into safety critical control systems and their robust development. To gain a deeper understanding of the process of designing and implementing such a system, a 2D pen plotter will be designed and a simulation of the robot will be developed using some of the techniques most used for safety critical systems implementation.
	
	\section{Motivations}
		\label{sec:intro_motivation} 
		
		The motivations of this thesis were to gain greater understanding of safety critical systems and the methods in which they are developed. This will include the methodology, design, testing and general implementation of a safety critical system. Unlike most software, safety critical systems require far more thought in the design and testing phase due to the danger that could follow an incorrect result or bug. 
	
	\subsection{Objective}
		\label{sec:intro_objective} 
		
		This thesis will research the methods and systems used to decrease the chances of a catastrophic errors occuring, the ways to handle the errors or control the system should an unexpected error occur and fail safes should the system not find such an error.
		
	\section{Overview}  
		\label{sec:intro_overview} 
		
		The remainder of chapter \ref{chap:intro} outlines the document structure and a brief overview of each section.

	\section{Outline} 
		\label{sec:intro_outline} 
		
		The methodology used for this project is a modified form of scrum for individuals, and hence the main body of the thesis will be broken down into the individual sprints. The structure of the main sections of this work are as follows:
		
		\begin{description}	
		
			\item[$\bullet$ Sprint 1 - Background Research]\hfill
			
			This section is the first sprint of the project and outlines the reasearch made into the major components of the project.
			
			\item[$\bullet$ Sprint 2 - DXF to GCode]\hfill
			
			This section will discuss the methods and libraries used to convert basic DXF files into basic GCode, the fail-safes added for incompatible files and the testing done on each method.
			
			\item[$\bullet$ Sprint 3 - Rasterized image to GCode]\hfill
						
			This section will discuss the methods and libraries used to convert JPEG and PNG images into basic GCode, the fail-safes added for incompatible files and the testing done on each method.

			\item[$\bullet$ Sprint 4 - Simulation Engine]\hfill
						
			This section will discuss the methods and libraries used to create a simulation engine for the 2d pen plotter. The reasons why a simulation was created rather than a physical plotter and the difficulties encountered during this process.

			\item[$\bullet$ Sprint 5 - Integrate Previous Sections]\hfill
						
			This section will discuss the methods used to integrate the previous sections into one functioning system backend. This will be a command line based system, requiring basic system testing and some more in depth unit testing.

			\item[$\bullet$ Sprint 6 - GUI]\hfill
						
			This section will discuss the methods and libraries used to create a GUI to control the backend and display the results of the previous sections.

			\item[$\bullet$ Sprint 7 - Working Prototype]\hfill
						
			This section will discuss the methods and libraries used to convert JPEG and PNG images into basic GCode, the fail-safes added for incompatible files and the testing done on each method.

			\item[$\bullet$ Sprint 8 - System Testing]\hfill
						
			This section will discuss the methods and libraries used to convert JPEG and PNG images into basic GCode, the fail-safes added for incompatible files and the testing done on each method.
			
		\end{description}